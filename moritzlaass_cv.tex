\documentclass[11pt,a4paper,sans]{moderncv}

% ModernCV themes
\moderncvstyle{classic}
\moderncvcolor{blue}

% Character encoding
\usepackage[utf8]{inputenc}

% Adjust the page margins
\usepackage[scale=0.75]{geometry}

% Personal Data
\name{Moritz}{Laass}
\title{PhD Candidate \& Software Engineer}
\homepage{www.bgd.lrg.tum.de/team/moritz\_laass.html}
\social[linkedin]{moritzlaass}

% Content
\begin{document}
\makecvtitle

\section{Research Interests}
\cvitem{}{Spatial Computing, Probabilistic Data Structures, High-Performance Big Data Management, GPU Acceleration, and Computational Geometry.}


\section{Education}
\cventry{2020--Present}{Doctoral Candidate (Big Geospatial Data Management)}{Technical University of Munich (TUM)}{Munich}{}{Research focus on High-Performance Computing, Machine Learning, and Hardware-Accelerated Spatial Joins using RTX. Project lead on Neural Search for Satellite Imagery.}
\cventry{2014--2017}{M.Sc. Human-Computer Interaction}{Ludwig Maximilians Universität (LMU)}{Munich}{Grade: 1.0 (Thesis)}{Master's Thesis: "Trajectory Simplification using Persistence." Developed novel algorithms for movement data compression. Specialized in Brain-Computer Interfaces and Deep Learning.}
\cventry{2007--2011}{B.A. Process Design}{FHNW Hyperwerk}{Basel, Switzerland}{}{Focus on Interactive Systems and Software Engineering. Developed a real-time collaborative game engine and high-resolution imaging systems.}

\section{Publications}
\subsection{Peer-Reviewed Conference Papers}
\cvitem{2021}{\textbf{M. Laass}, M. Kiermeier, M. Werner. "Improving persistence based trajectory simplification." \textit{2021 22nd IEEE International Conference on Mobile Data Management (MDM)}, pp. 157-162.}
\cvitem{2021}{\textbf{M. Laass}. "Point in Polygon Tests Using Hardware Accelerated Ray Tracing." \textit{Proceedings of the 29th International Conference on Advances in Geographic Information Systems (SIGSPATIAL)}.}
\cvitem{2021}{G. Dax, \textbf{M. Laass}, M. Werner. "Genetic Algorithm for Improved Transfer Learning Through Bagging Color-Adjusted Models." \textit{2021 IEEE International Geoscience and Remote Sensing Symposium (IGARSS)}, pp. 2612-2615.}
\cvitem{2021}{S. Götzer, \textbf{M. Laass}, G. Dax, M. Werner. "ObservaToriUM: A Simple Scalable Earth Observation Processing Engine." \textit{Symposium für Angewandte Geoinformatik (AGIT)}.}
\cvitem{2020}{M. Werner, G. Dax, \textbf{M. Laass}. "Computational challenges for artificial intelligence and machine learning in environmental research." \textit{INFORMATIK 2020}, pp. 1009-1017.}

\newpage
\section{Academic \& Teaching Experience}
\cventry{2020--2022}{Wissenschaftlicher Mitarbeiter (Research Associate)}{Technical University of Munich}{Munich}{}{
\begin{itemize}
    \item Conducted research in Big Geospatial Data Science and Distributed Systems.
    \item \textbf{Teaching:} Tutored students in Fundamentals of Computer Science and Geospatial Computing.
    \item \textbf{Mentorship:} Supervised student projects in Machine Learning, Change Detection, and High-Performance Computing.
    \item Provided technical training in Matlab, C++, and Algorithms \& Data Structures.
\end{itemize}}

\section{Professional Experience}
\cventry{2022--Present}{Fractional CTO \& AI Developer}{Freelance / Remote}{}{}{
\begin{itemize}
    \item Leading "AI-first" development initiatives using LLM Agents, RAG-Systems, and toolsets like Claude Code and Gemini CLI.
    \item Training hundreds of engineers in advanced AI integration and agentic workflows.
\end{itemize}}
\cventry{2020--2021}{Software Developer (Lead)}{Klang2 GmbH}{Munich}{}{Architected cross-platform mobile applications using Flutter and Python/Django.}
\cventry{2018--2020}{Backend Developer}{Motius GmbH}{Munich}{}{Built mission-critical microservices (Node.js, MongoDB, Redis) for IoT and SSO platforms.}
\cventry{2016--2016}{CTO \& Lead Developer}{e-bot7 GmbH}{Munich}{}{Defined the software architecture for an enterprise chatbot platform.}
\cventry{2014--2015}{CTO \& Lead Developer}{Soap Audio}{Munich / Tel-Aviv}{}{Co-founded a hardware startup; developed prototypes using embedded Linux and Android SDK.}

\section{Technical Skills}
\cvitem{Languages}{Python, C++, TypeScript, Go, C\#, Java, Rust, SQL, Bash.}
\cvitem{AI \& Data}{PyTorch, Scikit-learn, LLM Orchestration, RAG, Big Data Analytics.}
\cvitem{Systems}{GPU (RTX), Distributed Systems, Docker, Kubernetes, PostgreSQL/PostGIS, MongoDB.}
\cvitem{Web/Mobile}{React, Svelte, Node.js, Flutter, Three.js, FastAPI, Django.}

\section{Languages}
\cvitemwithcomment{German}{Native}{}
\cvitemwithcomment{English}{Fluent}{}

\end{document}